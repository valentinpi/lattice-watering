\documentclass[10pt, xcolor=svgnames]{beamer}

\usepackage[T1]{fontenc}
\usepackage[utf8]{inputenc}

\usetheme{Pittsburgh}
\usecolortheme{dove}

\title{Lattice Watering: Second Status Report}

\author{Christian Müller, Jonas Heinemann, Kaan Dönmez, Valentin Pickel}

\institute{
    Software Project on Internet Communication

    Summer Term 2022
    
    Freie Universität Berlin

    Institute for Computer Science
}

\date{\today{ }(newest version)}

\begin{document}

\maketitle

\begin{frame}{Updates}
    \begin{itemize}
        \item The NIB did not seem to work last week, but we finally got around it on the 1st of June. So we were able to send CoAP packets from the BR to the host and, naturally, from the BR to nodes. What did not work was sending a packet to the host. The network setup is now automated and the routes are properly configured. Additionally, we setup RPL.
        \item We added the use of GCoAP.
        \item We added DTLS support.
        \item We properly documented how the hardware is setup, especially how one can wire a node themselves.
        \item Added documentation on how to setup the hardware.
    \end{itemize}
\end{frame}

\begin{frame}{RIOT proves to be a bit limited}
    \begin{itemize}
        \item The border router setup is still very unintuitive and documentation for it is not very well written. At least there is some.
        \item WolfSSL is not supported for GCoAP, so we are limited here practically, since only using DTLS sockets makes the task harder.
        \item For TinyDTLS, the only allowed pseudorandom generators are \texttt{prng\_sha1prng}, \texttt{prng\_sha256prng} and \texttt{prng\_hwrng}, despite standardized ones existing. (see \texttt{prng\_tinymt32} from RFC8682)
        \item Many interfaces still seem to lack features, according to documentation. See e.g. \texttt{adc.h}.
    \end{itemize}
\end{frame}

\end{document}
