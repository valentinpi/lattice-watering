\documentclass[10pt, a4paper]{article}

\usepackage[T1]{fontenc}
\usepackage{fullpage}
\usepackage[utf8]{inputenc}
\usepackage{siunitx}
\usepackage{tikz}
\usepackage[european]{circuitikz}

\begin{document}
    \begin{center}
        \textsc{Lattice Watering}

        \vspace{\baselineskip}

        Handout

        \vspace{\baselineskip}

        Christian Müller, Jonas Heinemann, Kaan Dönmez, Valentin Pickel

        \vspace{\baselineskip}

        Software Project on Internet Communication
        
        Summer Term 2022

        Freie Universität Berlin

        \vspace{\baselineskip}

        \today{ }(latest version)
    \end{center}

    \begin{figure}[htbp!]
        \begin{center}
            \begin{circuitikz}
                \draw (0, 0) node[left] {\(\qty{92}{\milli\ampere}, \qty{3.8}{\volt}\)}
                to[R=$\qty{200}{\ohm}$, -*] (2, 0);
                \draw (2, 0)
                to[R=\(\qty{5}{\kilo\ohm}\)] (2, -2) node[ground] {};
                \draw (4, 0) node[nigfete, tr circle] (mos) {}
                (mos.source) node[anchor=north] {}
                (mos.gate) node[anchor=east] {}
                (mos.drain) node[anchor=south] {};
                \draw (mos.inner up) -- (mos.body E out);
                \draw (mos.body E out) -- (mos.body C out);
                \draw (mos.body C out) -- (mos.body C in);
                \draw (2, 0) -- (mos.gate);
                \draw (mos.source) -- (4, -2) node[ground] {};
                \draw (mos.drain) -- (4, 2) node[circle] {P};
            \end{circuitikz}
            \caption{Circuit for the pump P.}
        \end{center}
    \end{figure}

    Thus, we obtain:
    \[
        I_2 = \frac{\qty{5}{\volt}}{\qty{200}{\ohm}} = \qty{25}{\milli\ampere}
    \]
    And:
    \[
        I_3 = \frac{\qty{5}{\volt}}{\qty{200}{\ohm}} = \qty{25}{\milli\ampere}
    \]
\end{document}
